\documentclass[10pt, pdf, hyperref={unicode}]{beamer}
\usetheme{Singapore}
\usepackage[utf8]{inputenc}
\usepackage[english, russian]{babel}
\usepackage[T2A]{fontenc}
\usepackage{amsmath}
\usepackage{graphicx}


\title{Отчёт по заданию №4}
\author{Турсуналиев Чингиз, Серебрякова София, Сарджаев Мередкули, Ройтман Андрей}
\date{апрель 2020 г.}


\begin{document}
	\begin{frame}
		\titlepage
	\end{frame}

	\begin{frame}{Число всех поставок}
		\begin{figure}
			\includegraphics[scale=0.5]{"defects.png"}
			\centering
		\end{figure}
		Слева — поставки Harpy \& Co. Справа — Westeros Inc.
		Рыжим обозначена масса брака, розовым — масса качественной продукции.\\
		Общая масса производства почти не отличается.
		
	\end{frame}

	\begin{frame}{Суммарное число брака}		
		\begin{figure}
			\includegraphics[scale=0.5]{"all broken.png"}
			\centering
		\end{figure}
		
		Слева — поставки Harpy \& Co. Справа — Westeros Inc.
		Как видно, количество бракованых мечей у первых заметно меньше.
	\end{frame}

	\begin{frame}{Помесячное производство}
		\begin{figure}
			\includegraphics[scale=0.23]{"made and broken for each month.png"}
			\centering
		\end{figure}
	
		Сверху — поставки Harpy \& Co. Снизу — Westeros Inc. Рыжим обозначена масса брака, розовым — масса качественной продукции.
	\end{frame}

	\begin{frame}{О качестве производства}
		\begin{block}{}
			По диаграммам на предыдущем слайде видно, что количество брака у кампании Harpy \& Co снижается заметно сильнее, чем у Westeros Inc.
		\end{block}
	
		\begin{block}{}
			Значит, со временем качество производства у первой кампании выросло больше.
		\end{block}
	\end{frame}

	\begin{frame}{Заключение и выводы}
		\begin{itemize}
			\item Качество производства у Harpy \& Co возростает быстрее, чем у Westeros Inc.
			
			\item То есть количество брака в поставках будет у них меньше.
			
			\item Рекомендуется отдать предпочтение кампании Harpy \& Co.
		\end{itemize}
	\end{frame}
\end{document}